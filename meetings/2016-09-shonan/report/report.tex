\documentclass[a4paper]{article}
\usepackage{graphics}
\usepackage{shonan}

\begin{document}

% Generate a standard cover of an NII Shonan Meeting Report.
\SHONANno{201X-Y}
\SHONANtitle{A Guide for Authoring an\\ NII Shonan Meeting Report}
\SHONANauthor{%
Organizer 1\\
Organizer 2\\
Organizer 3}
\SHONANdate{January XX--YY, 201X}
\SHONANmakecover

\title{A Guide for Authoring an\\ NII Shonan Meeting Report}
\author{Organizers:\\
Organizer 1 (Affiliation 1)\\
Organizer 2 (Affiliation 2)\\
Organizer 3 (Affiliation 3)}
\date{January XX--YY, 201X}
\maketitle

NII publishes online an ``NII Shonan Meeting Report'' (ISSN 2186-7437)
for each NII Shonan Meeting.  The organizers of a meeting are
requested to contribute a report after their meeting.  A report
typically consists of the following materials:
\begin{itemize}
\item A cover in the standard format ({\em required\/});
\item The title of the meeting, the names of the organizers, and the
dates of the meeting (as shown above);
\item An overview of the meeting (as shown here);
\item A collection of the abstracts of the participants' talks (see
the page3);
\item A list of participants (see the page3).
\end{itemize}
Also, a report may include the following additional materials:
\begin{itemize}
\item A brief schedule of the meeting (see the page4);
\item A summary of talks;
\item A summary of discussions;
\item A summary of new findings;
\item A summary of identified issues.
\end{itemize}
However, these materials except the standard cover are not required.

The paper size needs to be A4.  Also, a standard \LaTeX{} style for
authoring an NII Shonan Meeting Report is provided.  (The organizers
may use a different format, but this \LaTeX{} style is still needed to
generate a standard cover.)  The rest of this guide introduces how to
use the standard \LaTeX{} style.

\clearpage

\section*{How to Use the Standard \LaTeX{} Style}
The file {\tt shonan.sty} defines the standard \LaTeX{} style.  Its
main role is to generate a standard cover.  The file {\tt shonan.eps}
is also needed to show the NII Shonan Meeting's logo on the cover.
The organizers can use the file {\tt report.tex} as a starting point
for their report.  Initially, it includes the content of this guide.

A cover is generated by the following commands:
\begin{verbatim}
\SHONANno{201X-Y}
\SHONANtitle{A Guide for Authoring an\\ NII Shonan Meeting Report}
\SHONANauthor{%
Organizer 1\\
Organizer 2\\
Organizer 3}
\SHONANdate{January XX--YY, 201X}
\SHONANmakecover
\end{verbatim}
The {\tt \textbackslash{}SHONANno} needs a report number as its
argument.  If the report describes the Y-th meeting in year 201X, the
report number is 201X-Y.  The {\tt \textbackslash{}SHONANtitle}, {\tt
  \textbackslash{}SHONANauthor}, and {\tt \textbackslash{}SHONANdate}
commands specify the title of the meeting, the names of the
organizers, and the dates of the meeting respectively.

\clearpage

% Use the \section command if you want to show the section number.
\section*{Overview of Talks}
\SHONANabstract{Title of Talk 1}{%
Speaker's Name, Speaker's Affiliation}
The abstract of Talk 1 appears here.

\SHONANabstract{Title of Talk 2}{%
Speaker's Name, Speaker's Affiliation}
The abstract of Talk 2 appears here.

\SHONANabstract{Title of Talk 3}{%
Speaker's Name, Speaker's Affiliation}
The abstract of Talk 3 appears here.

\bigskip

As shown above, a report may present a collection of talk abstracts.
The standard \LaTeX{} style provides the {\tt
  \textbackslash{}SHONANabstract} command to show the title and the
speaker's name and affiliation.  For each talk, put text in the
following form:
\begin{verbatim}
\SHONANabstract{Title of Talk}{%
Speaker's Name, Speaker's Affiliation}
The abstract of this talk appears here.
\end{verbatim}




\section*{List of Participants}
\begin{itemize}
\item Participant 1, Affiliation 1
\item Participant 2, Affiliation 2
\item Participant 3, Affiliation 3
\item Participant 4, Affiliation 4
\item Participant 5, Affiliation 5
\end{itemize}

\clearpage

\section*{Meeting Schedule}
\begin{bfseries}
Check-in Day: January XX (Sun)
\end{bfseries}
\begin{itemize}
\item Welcome Banquet
\end{itemize}
\begin{bfseries}
{Day1: January XX (Mon)}
\end{bfseries}
\begin{itemize}
\item Talks and Discussions
\item Group Photo Shooting
\end{itemize}
\begin{bfseries}
Day2: January XX (Tue)
\end{bfseries}
\begin{itemize}
\item Talks and Discussions
\end{itemize}
\begin{bfseries}
Day3: January XX (Wed)
\end{bfseries}
\begin{itemize}
\item Talks and Discussions
\item Excursion and Main Banquet
\end{itemize}
\begin{bfseries}
Day4: January XX (Thu)
\end{bfseries}
\begin{itemize}
\item Talks and Discussions
\item Wrap up
\end{itemize}
\end{document}
